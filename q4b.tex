% !TeX root = ./TMA02.tex
The Jacobi equation will be used to determine the nature of this stationary path as follows.\marginnote{HB p.22}[0cm]

Let $n > 1$ be a positive integer.
\[
	F\lr{y,y^\prime} = \lr{y^\prime}^n\e^y.
\]
\[
	P\lr{x} = \pderiv{^2F}{y^{\prime2}} = \pderiv{^2\lr{\lr{y^\prime}^n\e^y}}{y^{\prime^2}},
\]
\[
	\pderiv{F}{y^\prime} = n\lr{y^\prime}^{n-1}\e^y.
\]
\[
	P(x)=\pderiv{^2F}{y^{\prime2}} = n\lr{n-1}\lr{y^\prime}^{n-2}\e^y,
\]
where $y = n\ln\lr{cx + e^{\frac{1}{n}}}$. Therefore,
\[
	y^\prime = \frac{nc}{\lr{cx + e^{\frac{1}{n}}}},\qquad\textrm{where}\quad c=\e^{\frac{A}{n}} - \e^{\frac{1}{n}}.
\]
So,
\begin{align*}
	P(x) =& n(n-1)\lr{\dfrac{nc}{cx + \e^{\frac{1}{n}}}}^{n-2}\e^{ n\ln\lr{cx + e^{\frac{1}{n}}}},\\
		 =& n(n-1)\lr{\dfrac{nc}{cx + \e^{\frac{1}{n}}}}^{n-2}\lr{cx + e^{\frac{1}{n}}}^n,\\
		 =& n(n-1)\lr{\dfrac{nc}{cx + \e^{\frac{1}{n}}}}^{n}\frac{\lr{cx + e^{\frac{1}{n}}}^2\lr{cx + e^{\frac{1}{n}}}^n}{(nc)^2},\\
		 =& n(n-1)\frac{(nc)^n\lr{cx + e^{\frac{1}{n}}}^{2+n}}{(nc)^2\lr{cx + e^{\frac{1}{n}}}^{n}},\\
		 =& n(n-1)(nc)^{n-2}\lr{cx + e^{\frac{1}{n}}}^{2} > 0.
\end{align*}
Note: $c=\e^{\frac{A}{n}} - \e^{\frac{1}{n}} > 0$ because $A>1$, $n>1$, thus, $n(n-1) > 0$ and $nc>0$. The term $\lr{cx + e^{\frac{1}{n}}}^{2} > 0$ regardless of the  terms within the brackets as it is squared.

Thus, if the stationary path $y(x)$ is an extremum then it will be a local minimum.

\[
	Q(x) = \pderiv{^2F}{y^2} - \deriv{}{x}\lr{\psderiv{F}{y}{y^\prime}}.
\]
\[
	F = \lr{y^\prime}^n\e^y,\qquad \pderiv{F}{y}=\lr{y^\prime}^n\e^y,\qquad \pderiv{^2F}{y^2}=\lr{y^\prime}^n\e^y.
\]
\[
	\pderiv{F}{y^\prime} = n\lr{y^\prime}^{n-1}\e^y,\qquad \psderiv{F}{y}{y^\prime} = n\lr{y^\prime}^{n-1}\e^y.
\]
So,
\begin{align*}
	Q(x) =& \lr{y^\prime}^n\e^y - \deriv{}{x}\lr{n\lr{y^\prime}^{n-1}\e^y},\\
	=& \lr{y^\prime}^n\e^y - \lr{n\lr{y^\prime}^{n-1}y^\prime\e^y +\e^yn(n-1)\lr{y^\prime}^{n-2}y^\dprime},\\
	=& \lr{y^\prime}^n\e^y - \lr{n\lr{y^\prime}^{n}\e^y +\e^yn(n-1)\lr{y^\prime}^{n-2}y^\dprime},\\
	=& \e^y\lr{\lr{y^\prime}^n - n\lr{y^\prime}^{n} - n(n-1)\lr{y^\prime}^{n-2}y^\dprime},\\
	=& \e^y\lr{(1-n)\lr{y^\prime}^n - n(n-1)\lr{y^\prime}^{n-2}y^\dprime},\\
	=& \e^y\lr{\lr{y^\prime}^n\lrs{(1-n)-n(n-1)\lr{y^\prime}^{-2}y^\dprime}},\\
	=& \e^y\lr{\lr{y^\prime}^n\lrs{(1-n)-\frac{n(n-1)}{\lr{y^\prime}^{2}}y^\dprime}}.
\end{align*}
Now,
\[
	y^\prime = \frac{nc}{cx + \e^{\frac{1}{n}}},\qquad y^\dprime = -\frac{nc^2}{\lr{cx + \e^{\frac{1}{n}}}^2}.
\]
Substituting into the last equation for $Q(x)$ for $y^\prime$ and $y^\dprime$ gives,
\begin{align*}
	Q(x) =& \e^y\lr{\frac{(nc)^n}{\lr{cx + \e^{\frac{1}{n}}}^n}\lrs{(1-n)-\frac{n(n-1)}{\frac{nc^2}{\lr{cx + \e^{\frac{1}{n}}}^2}}\lr{-\frac{nc^2}{\lr{cx + \e^{\frac{1}{n}}}^2}}}},\\
	=& \e^y\lr{\frac{(nc)^n}{\lr{cx + \e^{\frac{1}{n}}}^n}\lrs{
			(1-n)-\frac{n(n-1)(-nc)^2}{(nc)^2}
	}},\\
	=& \e^y\lr{
			\frac{(nc)^n}{\lr{cx + \e^{\frac{1}{n}}}^n}\lrs{
				(1-n) + (n-1)
			}
	},\\
	=& 0.
\end{align*}
The Jacobi equation is\marginnote{HB p.22}[1.25cm]
\[
	\deriv{}{x}\lr{
		P(x)\deriv{u}{x}
	} - Q(x)u = 0,\quad	u(a) = 0,\quad u^\prime(a) = 1.
\]
Thus,
\[
	P(x)\deriv{u}{x} = k,\quad \textrm{where $k$ is a constant.}
\]
\[
	\deriv{u}{x} =k\frac{1}{P(x)} = k\cdot\frac{\lr{cx+\e^{\frac{1}{n}}}^{-2}}{n(n-1)(nc)^{n-2}},\quad n>1.
\]
\[
	\textrm{Redefining } k,\quad \deriv{u}{x} = k\lr{cx + \e^{\frac{1}{n}}}^{-2}.
\]
\[
	u = k\Int{•}{•}{x}\frac{1}{\lr{cx + \e^{\frac{1}{n}}}^2}.
\]
\[
	\textrm{Let } s = cx + \e^{\frac{1}{n}},\quad \textrm{then } \deriv{s}{x} = c,
\]
\[
	\textrm{and}\quad \deriv{x}{s} = \dfrac{1}{\deriv{s}{x}} = \frac{1}{c}.
\]
\[
\textrm{Thus, } u=k\Int{•}{•}{s} \frac{1}{s^2}\deriv{x}{s} = \frac{k}{c}\Int{•}{•}{s}s^{-2}.
\]
\[
	u = A - \frac{k}{c}s^{-1},\qquad\textrm{where $A$ is a constant.}
\]
Substituting back for $s$ gives, $u=A - \dfrac{k}{c}\dfrac{1}{\lr{cx + \e^{\frac{1}{n}}}}$.

Applying the initial condition $u(0) = 0$ enables the constant $A$ to be found,
\[
	A = \dfrac{k}{c\cdot\e^\frac{1}{n}}.
\]
Hence,
\[
	u(x) = \dfrac{k}{c\cdot\e^\frac{1}{n}} - \dfrac{k}{c}\dfrac{1}{\lr{cx + \e^{\frac{1}{n}}}}.
\]
Differentiating the above expression with respect to $x$ gives,
\[
	u^\prime(x) = (-1)\cdot\lr{-\frac{k}{c}}\cdot\frac{1}{\lr{cx + \e^{\frac{1}{n}}}^2}\cdot c = k\cdot \frac{1}{\lr{cx + \e^{\frac{1}{n}}}^2}.
\]
Applying the initial condition $u^\prime(0)=1$ gives,
\[
	1 = k\cdot \frac{1}{\lr{\e^\frac{1}{n}}^2},\quad \textrm{thus, } k = \e^{\frac{2}{n}}.
\]
Hence,
\begin{align*}
	u(x) = & \frac{k}{c \e^{\frac{1}{n}}} - \frac{k}{c} \frac{1}{\lr{cx + \e^{\frac{1}{n}}}},\quad\textrm{where, } k=\e^{\frac{2}{n}},\\
	u(x) = & \frac{\e^{\frac{2}{n}}}{c \e^{\frac{1}{n}}} - \frac{\e^{\frac{2}{n}}}{c} \frac{1}{\lr{cx + \e^{\frac{1}{n}}}},\\
	u(x) = & \frac{\e^{\frac{1}{n}}}{c} - \frac{\e^{\frac{2}{n}}}{c} \frac{1}{\lr{cx + \e^{\frac{1}{n}}}},
\end{align*}
\begin{align*}
	u(x) = & \frac{\e^{\frac{1}{n}}}{c}\lrs{1 - \frac{\e^{\frac{1}{n}}}{\lr{cx + \e^{\frac{1}{n}}}}},\\
	u(x) = & \frac{\e^{\frac{1}{n}}}{c}\lrs{\frac{cx + \e^{\frac{1}{n}} - \e^{\frac{1}{n}}}{\lr{cx + \e^\frac{1}{n}}}},\\
	u(x) = & \frac{\e^{\frac{1}{n}}}{c}\lrs{\frac{cx}{\lr{cx + \e^\frac{1}{n}}}},\\
	u(x) = & \frac{\e^\frac{1}{n} x}{cx + \e^\frac{1}{n}}.
\end{align*}
Equating $u(x)$ to zero gives,
\begin{align*}
	0 =& \frac{e^\frac{1}{n} x}{cx + \e^\frac{1}{n}},\\
	0 =& \e^\frac{1}{n} x,
\end{align*}
which can only be true when $x=0$.  Therefore, the closed interval $[0,1]$ does not contain points conjugate to the point $x=0$. 

From Theorem~8.4:\marginnote{A sufficient condition, Notes, p.185.}[0cm]
\begin{itemize}
  \item The function $y(x)$ satisfies the \el;
  \item Along the curve $y(x), P(x) = F_{y^\prime y^\prime} > 0$ for $0\leq x \leq 1$; and
  \item The closed interval $[0,1]$ contains no points conjugate to the point $x=0$.
\end{itemize}
Hence, the functional has a weak minimum along $y(x)$.
