% !TeX root = ./TMA02.tex
Given
\[
	S[y] = \int_a^b \dd x\left(y^2\sinh x -\frac{2y^{\prime 2}}{\sinh x}\right),\quad y(a) = A,\quad y(b) = B,\quad 0< a < b,
\]
then
\[
	\textrm{Let } F = y^2\sinh x -\frac{2y^{\prime 2}}{\sinh x}
\]
The following derivatives are required to determine the \el\marginnote{Making use of the quotient rule here.}
\begin{equation*}
\begin{split}
	\pderiv{F}{y^\prime} =& -\frac{4y^\prime}{\sinh x}.\\\\
	\deriv{}{x}\left(\pderiv{F}{y^\prime}\right) =& \deriv{}{x}\left( -\frac{4y^\prime}{\sinh x}\right),\\
	=& \frac{\sinh x\left(-4y^\dprime\right)-\left(-4y^\prime\right)\cosh x }{\sinh^2x},\\
	=& \frac{4}{\sinh x}\left(y^\prime\frac{\cosh x}{\sinh x}-y^\dprime\right).\\\\
	\textrm{and}\quad\quad{} \pderiv{F}{y}=& 2y\sinh x.
\end{split}
\end{equation*}
The \el is given by
\[
	\EL,
\]
and substituting into this equation the expressions for the derivatives obtained above give the \el associated with the functional $S[y]$,
\[
	\frac{4}{\sinh x}\left(y^\prime\frac{\cosh x}{\sinh x}-y^\dprime\right) - 2y\sinh x = 0,
\]
which can be rearranged into the following,
\begin{gather}
	2\sinh(x)\cdot y^\dprime - 2\cosh(x)\cdot y^\prime + \sinh^3(x)\cdot y = 0, \\ 
	y(a) = A,\, y(b) = B,\, 0< a < b.\nonumber
\end{gather}